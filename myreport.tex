% \documentclass{myreport}% 选项 forprint: 交付打印时添加, 避免彩色链接字迹打印偏淡. 即使用下一行:
% \documentclass[forprint]{myreport}
\documentclass{myreport}

\usepackage{tcolorbox}
\usepackage{listings}
% \usepackage{float}
\usepackage[section]{placeins}


\lstdefinestyle{lfonts}{
    basicstyle = \footnotesize\ttfamily,
    stringstyle = \color{purple},
    keywordstyle = \color{blue!60!black}\bfseries,
    commentstyle = \color{olive}\scshape,
}
\lstdefinestyle{lnumbers}{
    numbers = left,
    numberstyle = \tiny,
    numbersep = 1em,
    firstnumber = 1,
    stepnumber = 1,
}
\lstdefinestyle{llayout}{
    breaklines = true,
    tabsize = 2,
    columns = flexible,
}
\lstdefinestyle{lgeometry}{
    xleftmargin = 20pt,
    xrightmargin = 0pt,
    frame = tb,
    framesep = \fboxsep,
    framexleftmargin = 20pt,
}
\lstdefinestyle{lgeneral}{
    style = lfonts,
    style = lnumbers,
    style = llayout,
    style = lgeometry,
}
\lstdefinestyle{c}{
	language = {c},
	style = lgeneral,
}
\setcounter{tocdepth}{3}

\begin{document}
%%%%%%%%%%%%%%%%%%%%%%%%%%%%%%%%%%%%%%%%%%%%%%%%%%%%%%%%%%%%%%%%%%%%%%%%%%%%%
% 封面
%%%%%%%%%%%%%%%%%%%%%%%%%%%%%%%%%%%%%%%%%%%%%%%%%%%%%%%%%%%%%%%%%%%%%%%%%%%%%
\title{计算机网络实验四:\\ 跨路由器实现VLAN}% TODO:标题
\Cschoolname{数据科学与计算机学院}          % 学院名
\Cmajor{计算机科学与技术}                  % 专业中文名
\StudentNumber{16337237,16337269,16337271} % 填写自己的学号
\author{王永锋,颜彬,杨陈泽}                            % 作者名字
\Csupervisor{陈立文}    %指导教师中文名、职称
\date{二〇一八年五月十七日} % TODO:               % 日期, 要注意和英文日期一致!!
\pdfbookmark[0]{封面}{title}         % 封面页加到 pdf 书签
\maketitle
\frontmatter
%%%%%%%%%%%%%%%%%%%%%%%%%%%%%%%%%%%%%%%%%%%%%%%%%%%%%%%%%%%%%%%%%%%%%%%%%%%%%
% 目录
%%%%%%%%%%%%%%%%%%%%%%%%%%%%%%%%%%%%%%%%%%%%%%%%%%%%%%%%%%%%%%%%%%%%%%%%%%%%%
% 把目录加入到书签
\pagenumbering{Roman}              % 正文之前的页码用大写罗马字母编号.
\pdfbookmark[0]{目录}{toc}
\tableofcontents
%% 以下是正文
\mainmatter 
%%%%%%%%%%%%%%%%%%%%%%%%%%%%%%%%%%%%%%%%%%%%%%%%%%%%%%%%%%%%%%%%%%%%%%%%%%%%
% 正文
%%%%%%%%%%%%%%%%%%%%%%%%%%%%%%%%%%%%%%%%%%%%%%%%%%%%%%%%%%%%%%%%%%%%%%%%%%%%
\chapter{小组成员及分工}

\section{小组成员}

\begin{table}[htp]
  \caption{小组成员信息}
  \centering
  \rowcolors{1}{White}{Lavender}
  \begin{tabular}{cc}
  \toprule
  组员姓名 & 学号 \\
  \midrule
  王永锋(组长) & 16337237 \\
  颜彬 & 16337269 \\
  杨陈泽 & 16337271 \\
  \bottomrule
  \hiderowcolors
\end{tabular}
\label{tab:group}
\end{table}

\section{小组分工表及自评}
\begin{table}[htp]
  \caption{小组分工表}
  \centering
  \rowcolors{1}{White}{Lavender}
  \begin{tabular}{lcp{11cm}}
    \toprule
    小组成员姓名 & 自评 & 分工 \\
    \midrule
    王永锋 & 100 & 实验6-2 6-3 操作实践及报告编写 及最终实验报告排版 \\
    颜彬 & 100 & 实验6-2 6-3操作实践 实验6-2报告编写,  \\
    杨陈泽 & 100 & 实验6-2 6-3操作实践 实验补充问题三 实验及报告编写 \\
  \bottomrule
  \hiderowcolors
  \end{tabular}
  \label{tab:group-devide}
\end{table}
%%%%%%%%%%%%%%%%%%%%%%%%%%%%%%%%%%%%%%%%%%%%%%%%%%%%%%%%%%%%%%%%%%%%%%%%%%%%
% 实验一:跨交换机的VLAN
%%%%%%%%%%%%%%%%%%%%%%%%%%%%%%%%%%%%%%%%%%%%%%%%%%%%%%%%%%%%%%%%%%%%%%%%%%%%
\chapter{实验一:跨交换实现VLAN}

\section{实验拓扑}

%\usepackage{changepage}
%\usepackage{rotating}
%\usepackage{float}
%\usepackage[section]{placeins}
%\begin{sidewaystable}[!Htp]
\begin{figure}[htp]
    %\begin{adjustwidth}{-1.5cm}{-1cm}
    \centering
    \includegraphics[width=13cm]{"./figure/2018-05-18-03-26-52.png"}
    \caption{实验拓扑}
    \label{fig:e1-topo}
    %\end{adjustwidth}
\end{figure}



\section{实验步骤}

\subsection{步骤1}

将我们的三台电脑按照拓扑图配置好IP地址,用网线将各电脑与交换机对应的端口相连。

然后检测三台电脑间能否互相ping通,下\autoref{fig:e1-s1-ping}展示了PC1 \texttt{ping} PC2 与PC3的结果,都能够ping通,表明网络线路没有问题。

%\usepackage{changepage}
%\usepackage{rotating}
%\begin{sidewaystable}[htp]
\begin{figure}[htp]
    %\begin{adjustwidth}{-1.5cm}{-1cm}
    \centering
    \includegraphics[width=13cm]{"./figure/2018-05-04-08-53-24.png"}
    \caption{PC1 ping PC2 PC3}
    \label{fig:e1-s1-ping}
    %\end{adjustwidth}
\end{figure}

\subsection{步骤2:路由器1设置vlan10}

该步骤需要路由器1将PC1对应的端口(gig 0/5)设置为vlan 10。

相关设置的步骤可见\autoref{fig:e1-s2-10}。

%\usepackage{changepage}
%\usepackage{rotating}
%\begin{sidewaystable}[htp]
\begin{figure}[htp]
    %\begin{adjustwidth}{-1.5cm}{-1cm}
    \centering
    \includegraphics[width=13cm]{"./figure/2018-05-04-09-10-21.png"}
    \caption{路由器1设置PC1端口为vlan 10}
    \label{fig:e1-s2-10}
    %\end{adjustwidth}
\end{figure}

完成设置后,由于PC1在VLAN 10中,而PC2,PC3在默认的VLAN1中,如果设置成功的话是PC2是无法ping通PC1的,下面进行验证。

\begin{enumerate}
    \item 使用指令\texttt{show vlan id 10}查看vlan10对应的端口,可见\autoref{fig:e1-s2-check-1}
    %\usepackage{changepage}
    %\usepackage{rotating}
    %\begin{sidewaystable}[htp]
    \begin{figure}[htp]
        %\begin{adjustwidth}{-1.5cm}{-1cm}
        \centering
        \includegraphics[width=13cm]{"./figure/2018-05-04-09-10-21.png"}
        \caption{查看VLAN 10对应的端口}
        \label{fig:e1-s2-check-1}
        %\end{adjustwidth}
    \end{figure}
    \item PC2 无法ping通PC1,可见\autoref{fig:e1-s2-check-ping}.
    %\usepackage{changepage}
    %\usepackage{rotating}
    %\begin{sidewaystable}[htp]
    \begin{figure}[htp]
        %\begin{adjustwidth}{-1.5cm}{-1cm}
        \centering
        \includegraphics[width=13cm]{"./figure/id001.png"}
        \caption{PC2 无法 ping 通 PC1}
        \label{fig:e1-s2-check-ping}
        %\end{adjustwidth}
    \end{figure}
\end{enumerate}


\section{步骤3:设置路由器1设置 VLAN 20}

该步骤要求将PC2对应的交换机的端口设置为VLAN 20。这时候,原本能够相互ping通的PC2与PC3也开始不能相互ping通。

相关设置的步骤可见\autoref{fig:e1-s3-vlan-20}.

%\usepackage{changepage}
%\usepackage{rotating}
%\begin{sidewaystable}[htp]
\begin{figure}[!htp]
    %\begin{adjustwidth}{-1.5cm}{-1cm}
    \centering
    \includegraphics[width=13cm]{"./figure/2018-05-04-09-18-39.png"}
    \caption{设置VLAN 20}
    \label{fig:e1-s3-vlan-20}
    %\end{adjustwidth}
\end{figure}

设置后,我们进行验证,查看路由器的配置与检查PC2与PC3是否能够ping通。

\begin{enumerate}
    \item 使用指令\texttt{show vlan id 20}查看vlan20对应的端口,可见\autoref{fig:e1-s3-check-1}
    %\usepackage{changepage}
    %\usepackage{rotating}
    %\begin{sidewaystable}[htp]
    \begin{figure}[!htp]
        %\begin{adjustwidth}{-1.5cm}{-1cm}
        \centering
        \includegraphics[width=13cm]{"./figure/2018-05-04-09-18-49.png"}
        \caption{查看VLAN 20对应的端口}
        \label{fig:e1-s3-check-1}
        %\end{adjustwidth}
    \end{figure}
    \item PC2 无法ping通PC3,可见\autoref{fig:e1-s3-check-ping}.
    %\usepackage{changepage}
    %\usepackage{rotating}
    %\begin{sidewaystable}[htp]
    \begin{figure}[!htp]
        %\begin{adjustwidth}{-1.5cm}{-1cm}
        \centering
        \includegraphics[width=13cm]{"./figure/id002.png"}
        \caption{PC2 无法 ping 通 PC3}
        \label{fig:e1-s3-check-ping}
        %\end{adjustwidth}
    \end{figure}
\end{enumerate}

\section{步骤4::设置路由器1端口的trunk模式}

该步骤要求设置两路由器相连的端口(0/24)设置为\texttt{trunk}模式。设置步骤可见\autoref{fig:e1-s4-trunk}。

%\usepackage{changepage}
%\usepackage{rotating}
%\begin{sidewaystable}[htp]
\begin{figure}[!htp]
    %\begin{adjustwidth}{-1.5cm}{-1cm}
    \centering
    \includegraphics[width=13cm]{"./figure/2018-05-04-09-23-00.png"}
    \caption{将路由器1的0/24端口设置为trunk模式}
    \label{fig:e1-s4-trunk}
    %\end{adjustwidth}
\end{figure}

设置完成后,使用\texttt{show interface gig 0/24 switchport}该指令验证,可见\autoref{fig:e1-s4-check}。

%\usepackage{changepage}
%\usepackage{rotating}
%\begin{sidewaystable}[htp]
\begin{figure}[!htp]
    %\begin{adjustwidth}{-1.5cm}{-1cm}
    \centering
    \includegraphics[width=13cm]{"./figure/2018-05-04-09-23-29.png"}
    \caption{验证路由器1的trunk设置}
    \label{fig:e1-s4-check}
    %\end{adjustwidth}
\end{figure}


这个时候由于路由器2上的端口尚未设置,因此仍然不能够ping通。

\section{步骤5:设置路由器2}

由于设置路由器2的指令与设置路由器1的指令类似,这里并不放设置的截图,仅放验证的截图。

由\autoref{fig:e1-s5-vlan20}可知,该次设置成功。

%\usepackage{changepage}
%\usepackage{rotating}
%\begin{sidewaystable}[htp]
\begin{figure}[htp]
    %\begin{adjustwidth}{-1.5cm}{-1cm}
    \centering
    \includegraphics[width=13cm]{"./figure/2018-05-04-10-12-46.png"}
    \caption{验证路由器2的vlan20设置}
    \label{fig:e1-s5-vlan20}
    %\end{adjustwidth}
\end{figure}

此时PC2,PC3之间仍不能够ping通。

\section{步骤6:设置路由器2端口的trunk模式}

路由器2的\texttt{gigabitethernet 0/24}端口与路由器1之间相连,需要设置为\texttt{trunk}模式用以转发链路帧,设置后,验证设置如\autoref{fig:e1-s6-trunk},可知设置成功。

%\usepackage{changepage}
%\usepackage{rotating}
%\begin{sidewaystable}[htp]
\begin{figure}[htp]
    %\begin{adjustwidth}{-1.5cm}{-1cm}
    \centering
    \includegraphics[width=13cm]{"./figure/2018-05-04-10-13-08.png"}
    \caption{验证PC2的trunk模式设置}
    \label{fig:e1-s6-trunk}
    %\end{adjustwidth}
\end{figure}

\section{步骤7:验证}

这一个步骤我们需要验证PC2与PC3之间能相互通信,但PC1与PC3不能相互通信。

因此我们完成了以下几个问题

%\usepackage{tcolorbox}
%\newtcolorbox{mybox}{}
%\renewtcolorbox{mybox}{colback = red!25!white, colframe = red!75!black}
%\begin{mybox}[title = {}]
\begin{tcolorbox}[title = {观察一}]
主机之间能否互相通信?
\end{tcolorbox}

PC1 不能ping通PC2,PC3, PC2,3之间可以相互ping通。

可见\autoref{fig:e1-s7-ping-suc}。



%\usepackage{changepage}
%\usepackage{rotating}
%\usepackage{float}
%\usepackage[section]{placeins}
%\begin{sidewaystable}[!Htp]
\begin{figure}[htp]
    %\begin{adjustwidth}{-1.5cm}{-1cm}
    \centering
    \includegraphics[width=13cm]{"./figure/2018-05-18-03-31-11.png"}
    \caption{PC2与PC3能够ping通}
    \label{fig:e1-s7-ping-suc}
    %\end{adjustwidth}
\end{figure}


%\usepackage{tcolorbox}
%\newtcolorbox{mybox}{}
%\renewtcolorbox{mybox}{colback = red!25!white, colframe = red!75!black}
%\begin{mybox}[title = {}]
\begin{tcolorbox}[title = {观察二}]
    能否检测到PC1,PC2,PC3的ICMP包?
\end{tcolorbox}
当一台主机能ping到另一台主机时,wireshark可以捕捉到另一台主机的ICMP response 包。\\ 
在本拓扑中, PC3可以捕捉到PC2的ICMP包,但无法捕捉到PC1的ICMP包。PC2可以捕捉到
PC3可以捕捉到PC2的ICMP包,但无法捕捉到PC1的ICMP包。PC2可以捕捉到PC3的ICMP包,但无法捕捉到PC1的ICMP包。
PC1由于不与PC2和PC3在同一个VLAN中,故其无法捕捉到PC2和PC3的ICMP包。

%\usepackage{tcolorbox}
%\newtcolorbox{mybox}{}
%\renewtcolorbox{mybox}{colback = red!25!white, colframe = red!75!black}
%\begin{mybox}[title = {}]
\begin{tcolorbox}[title = {观察三}]
能否捕获到Trunk链路上的VLAN ID?请讨论原因。
\end{tcolorbox}
交换机的vlan模式分为access,trunk和hybrid三种。\\
在trunk模式的接收端口中,所有进入的链路层帧,有tag时,不改变原来的tag,直接
转发该帧;若帧无tag,则加上PVID(端口默认VLAN的ID,默认为1)和加上默认优先级,并转发。\\
在trunk模式的发送端口中,若帧带有tag(且VID与PVID不同),则不修改tag直接转发该包。若
帧不带有tag,则直接进行转发。\\ 

在access模式的发送端口中,若VID等于PVID,则去掉tag再转发。若VID不等于PVID,则不转发。

综上讨论,所有主机不可能捕获到Trunk链路上的VID。这是因为所有主机通过access模式
连接到路由器。所有链路层帧要么被access模式的端口丢弃,要么被去除tag。所有无法捕获到VID。

%\usepackage{changepage}
%\usepackage{rotating}
%\usepackage{float}
%\usepackage[section]{placeins}
%\begin{sidewaystable}[!Htp]
\begin{figure}[htp]
    %\begin{adjustwidth}{-1.5cm}{-1cm}
    \centering
    \includegraphics[width=13cm]{"./figure/194156585.png"}
    \caption{补充说明}
    \label{fig:e1}
    %\end{adjustwidth}
\end{figure}


%\usepackage{tcolorbox}
%\newtcolorbox{mybox}{}
%\renewtcolorbox{mybox}{colback = red!25!white, colframe = red!75!black}
%\begin{mybox}[title = {}]
\begin{tcolorbox}[title = {观察四}]
    
    \begin{itemize}
        \item 查看交换机的地址表。清楚地址表,适当更改、增加网线接口,然后观察,分析地址表的形成与变化过程(配合wireshark分析泛洪现象)。
        \item show mac-address-table命令显示的MAC地址与在命令提示符下通过\texttt{ifconfig /all}命令显示的 \texttt{mac}地址是否相同。
    \end{itemize}
    
\end{tcolorbox}
以下讨论基于下列假设:
A主机 192.168.10.200
B主机 192.168.10.20
两个主机共同处于10号VLAN中。
展示的wireshark截图中,wireshark均运行在B主机。

在交换机刚启动时,其mac地址表为空。当一个帧经过交换机时,交换机会提取源mac地址。若源mac地址已经在地址表中,
则该表项的失效时间被刷新(重新回到300s)。否则,交换机将源mac地址和输入端口这一对信息存如MAC地址表中,并设置失效时间到300s。
每当一个帧进入交换机时,交换机会检查其目的MAC地址。若地址在MAC地址表中,则交换机根据地址表的信息将帧送到输出端口。
若该地址不在表中,则交换机将帧送出到每一个端口。特别地,由于没有一个源MAC地址是广播地址(`FF:FF:FF:FF:FF:FF`),
故广播地址不可能出现在MAC地址表中(不可能被交换机学习到)。故所有以广播地址为目的地址的帧,都会被广播。
下图显示了A主机ping向B主机时,ARP请求、ARP应答和ACMP协议的抓包情况。\\ 

%\usepackage{changepage}
%\usepackage{rotating}
%\usepackage{float}
%\usepackage[section]{placeins}
%\begin{sidewaystable}[!Htp]
\begin{figure}[htp]
    %\begin{adjustwidth}{-1.5cm}{-1cm}
    \centering
    \includegraphics[width=13cm]{"./figure/ws01.png"}
    \caption{}
    \label{fig:e1-ip}
    %\end{adjustwidth}
\end{figure}


每当主机更换IP地址时,它都会告知其所有邻居。如\autoref{fig:e1-ip}所示。

假设主机A将其IP地址更换成192.168.10.111。主机A会首先使用ARP协议请求自己的IP地址对应的mac地址。
这一步的目的是确保自己的IP不与子网内的其他IP地址冲突。当IP地址不冲突时,该ARP请求应该得不到任何回应。
主机A在等待一段时间且没有接收到ARP应答时,可以确信自己的IP地址不重复,于是主机A发送Gratuitous ARP帧。
Gratuitous ARP是一种主动ARP请求,它的作用是告知其他主机,本主机的IP地址和MAC地址,省去其他主机发送ARP请求的麻烦。
Gratuitous ARP是一种ARP请求,但该请求不会收到任何应答。

%\usepackage{changepage}
%\usepackage{rotating}
%\usepackage{float}
%\usepackage[section]{placeins}
%\begin{sidewaystable}[!Htp]
\begin{figure}[htp]
    %\begin{adjustwidth}{-1.5cm}{-1cm}
    \centering
    \includegraphics[width=13cm]{"./figure/ws02.png"}
    \caption{补充说明}
    \label{fig:e1-arp}
    %\end{adjustwidth}
\end{figure}



%\usepackage{tcolorbox}
%\newtcolorbox{mybox}{}
%\renewtcolorbox{mybox}{colback = red!25!white, colframe = red!75!black}
%\begin{mybox}[title = {}]
\begin{tcolorbox}[title = {观察五}]
判断实验是否达到预期目标。
\end{tcolorbox}
在本次实验中,我们所有的实验现象都符合我们的预期。\\ 

同一个交换机中,同一个VLAN内的主机可以ping通,不同VLAN内的主机无法
ping通。在不同交换机中,用trunk模式相连两个交换机时,同一个VLAN内的主机可以连通。若不设置trunk模式,则跨交换机时主机无法ping
通。\\ 

同时,我们还根据实验现象给出了思考和解释。VLAN的三种访问模式是access, trunk和hybrid.在本实验中我们涉及到的是
access(交换机与主机相连)和trunk(交换机之间互联)。这三种模式都对链路层帧的tag情况有不同的行为。所有的实验现象
都符合访问模式的特点。前面的题目中,我们给出了相关的解释。
\section{实验思考}

思考题来自于老师的pdf材料.

%\usepackage{tcolorbox}
%\newtcolorbox{mybox}{}
%\renewtcolorbox{mybox}{colback = red!25!white, colframe = red!75!black}
%\begin{mybox}[title = {}]
\begin{tcolorbox}[title = {思考题一}]
说明vlan技术中的trunk模式端口的意义。
\end{tcolorbox}
如图\autoref{fig:e1}所示。\\ 
配置为trunk模式的接收端口中,若其接收到不带tag的帧,则自动为其加上PVID(Port VLAN ID)
和默认优先级。否则,该帧的tag不被修改,直接进行转发。\\ 
配置为trunk模式的发送端口中,若帧带有tag的VLAN ID恰好为PVID,则去掉VLAN ID tag再转发。否则,
不改变tag直接转发。\\ 
其中PVID(Port VLAN ID)是端口的默认VLAN ID。若没有被配置,trunk端口的PVID为VLAN 1。

\begin{tcolorbox}[title = {思考题二}]
如何查看trunk接口允许哪些VLAN通过?
\end{tcolorbox}

可以使用\texttt{show interface gig 0/24 switchport}命令来查看,效果如下图所示。其中红箭头表示允许的vlan为所有(all)。


%\usepackage{changepage}
%\usepackage{rotating}
%\usepackage{float}
%\usepackage[section]{placeins}
%\begin{sidewaystable}[!Htp]
\begin{figure}[htp]
    %\begin{adjustwidth}{-1.5cm}{-1cm}
    \centering
    \includegraphics[width=13cm]{"./figure/2018-05-18-02-31-27.png
    "}
    \caption{查看允许VLAN}
    \label{fig:e1-t2}
    %\end{adjustwidth}
\end{figure}


\begin{tcolorbox}[title = {思考题三}]
实验开始前请先确定三台PC机处于一个网段里面。为什么做这样的限定?
\end{tcolorbox}

我们这一次实验并没有设计默认网关,也没有设置路由转发,因此当发往处于不同网段的主机时,不可能发送过去。


\chapter{实验二:通过三层交换机实现VLAN间路由}

\section{实验拓扑}

%\usepackage{changepage}
%\usepackage{rotating}
%\usepackage{float}
%\usepackage[section]{placeins}
%\begin{sidewaystable}[!Htp]
\begin{figure}[htp]
    %\begin{adjustwidth}{-1.5cm}{-1cm}
    \centering
    \includegraphics[width=13cm]{"./figure/2018-05-18-03-25-45.png"}
    \caption{实验拓扑图}
    \label{fig:e2-topo}
    %\end{adjustwidth}
\end{figure}


\section{实验步骤}

分析:本实验的预期是将\autoref{fig:e2-topo}中的三台计算机,划分进不同的VLAN,并让处于不同VLAN的计算机互相隔离.然后启用三层交换机的路由功能,让已经隔离的计算机能互相通信.
(如:隔离后PC1能ping通PC2,PC3).

\subsection{步骤1:连接线路并测试连通性}

\begin{enumerate}
    \item 设置每一台主机的IP地址
    %\usepackage{changepage}
    %\usepackage{rotating}
    %\usepackage{float}
    %\usepackage[section]{placeins}
    %\begin{sidewaystable}[!Htp]
    \begin{figure}[htp]
        %\begin{adjustwidth}{-1.5cm}{-1cm}
        \centering
        \includegraphics[width=13cm]{"./figure/2018-05-17-22-31-49.png"}
        \caption{PC1设置IP地址的截图}
        \label{fig:e2-s2-set-ip}
        %\end{adjustwidth}
    \end{figure}
    
    \item 测试PC1, PC2, PC3的连通性,发现PC1无法ping通PC2,PC3,其他的PC2和PC3可以相互ping通。
    %\usepackage{tcolorbox}
    %\newtcolorbox{mybox}{}
    %\renewtcolorbox{mybox}{colback = red!25!white, colframe = red!75!black}
    %\begin{mybox}[title = {}]
    %\usepackage{changepage}
    %\usepackage{rotating}
    %\usepackage{float}
    %\usepackage[section]{placeins}
    %\begin{sidewaystable}[!Htp]
    \begin{figure}[htp]
        %\begin{adjustwidth}{-1.5cm}{-1cm}
        \centering
        \includegraphics[width=13cm]{"./figure/2018-05-17-22-32-57.png"}
        \caption{PC1无法ping通PC2,PC3}
        \label{fig:e2-s2-not-ping}
        %\end{adjustwidth}
    \end{figure}
    
    \begin{tcolorbox}[title = {思考}]
    PC1的网段不同于PC2,PC3,请讨论原因
    \tcblower
    这里做的是不同vlan间通过路由转发消息的实验。如果PC1与PC2,PC3所在网段相同,那么PC1在发送IP包的时候就会认为在同一个子网中,永远也不会发往默认网关。
    \end{tcolorbox}
    \item 使用\texttt{show ip route}命令查看三层交换机的路由表,并记录
    %\usepackage{changepage}
    %\usepackage{rotating}
    %\usepackage{float}
    %\usepackage[section]{placeins}
    %\begin{sidewaystable}[!Htp]
    \begin{figure}[htp]
        %\begin{adjustwidth}{-1.5cm}{-1cm}
        \centering
        \includegraphics[width=13cm]{"./figure/2018-05-17-17-00-58.png"}
        \caption{查看三层交换机的路由表}
        \label{fig:e2-s1-route}
        %\end{adjustwidth}
    \end{figure}
    

\end{enumerate}

\subsection{步骤2:交换机A创建VLAN10}

该步骤需要在交换机A上创建VLAN10,并将端口0/5(即PC1对应的接口)划分到VLAN10中.

%\usepackage{changepage}
%\usepackage{rotating}
%\usepackage{float}
%\usepackage[section]{placeins}
%\begin{sidewaystable}[!Htp]
\begin{figure}[htp]
    %\begin{adjustwidth}{-1.5cm}{-1cm}
    \centering
    \includegraphics[width=13cm]{"./figure/2018-05-17-17-03-01.png"}
    \caption{交换机A创建VLAN 10}
    \label{fig:e2-s2-vlan10}
    %\end{adjustwidth}
\end{figure}


操作完成后,使用\texttt{show vlan id 10}验证实验操作.

%\usepackage{changepage}
%\usepackage{rotating}
%\usepackage{float}
%\usepackage[section]{placeins}
%\begin{sidewaystable}[!Htp]
\begin{figure}[htp]
    %\begin{adjustwidth}{-1.5cm}{-1cm}
    \centering
    \includegraphics[width=13cm]{"./figure/2018-05-17-17-03-09.png"}
    \caption{查看vlan 10信息}
    \label{fig:e2-s2-see-vlan10}
    %\end{adjustwidth}
\end{figure}




\subsection{步骤3:交换机A创建VLAN20}

该步骤需要在交换机A上创建VLAN20,并将端口0/15(即PC2对应端口)划分到VLAN20中.

%\usepackage{changepage}
%\usepackage{rotating}
%\usepackage{float}
%\usepackage[section]{placeins}
%\begin{sidewaystable}[!Htp]
\begin{figure}[htp]
    %\begin{adjustwidth}{-1.5cm}{-1cm}
    \centering
    \includegraphics[width=13cm]{"./figure/2018-05-17-17-03-55.png"}
    \caption{创建VLAN20}
    \label{fig:e2-s3-vlan20}
    %\end{adjustwidth}
\end{figure}



操作完成后,使用\texttt{show vlan id 20}验证实验操作.

%\usepackage{changepage}
%\usepackage{rotating}
%\usepackage{float}
%\usepackage[section]{placeins}
%\begin{sidewaystable}[!Htp]
\begin{figure}[htp]
    %\begin{adjustwidth}{-1.5cm}{-1cm}
    \centering
    \includegraphics[width=13cm]{"./figure/2018-05-17-17-04-27.png"}
    \caption{查看VLAN20}
    \label{fig:e2-s3-see-vlan20}
    %\end{adjustwidth}
\end{figure}



\subsection{步骤4:设置A与B连接的端口模式}

将交换机A上与交换机B相连的端口(假设为端口0/24)定义为\texttt{tag VLAN}模式。设置步骤可见

%\usepackage{changepage}
%\usepackage{rotating}
%\usepackage{float}
%\usepackage[section]{placeins}
%\begin{sidewaystable}[!Htp]
\begin{figure}[htp]
    %\begin{adjustwidth}{-1.5cm}{-1cm}
    \centering
    \includegraphics[width=13cm]{"./figure/2018-05-17-17-05-54.png
    "}
    \caption{设置端口模式}
    \label{fig:e2-s4-trunk}
    %\end{adjustwidth}
\end{figure}



操作完成后,使用\texttt{show interface gig 0/24 switchport}验证设置.
%\usepackage{changepage}
%\usepackage{rotating}
%\usepackage{float}
%\usepackage[section]{placeins}
%\begin{sidewaystable}[!Htp]
\begin{figure}[htp]
    %\begin{adjustwidth}{-1.5cm}{-1cm}
    \centering
    \includegraphics[width=13cm]{"./figure/2018-05-17-17-06-02.png"}
    \caption{验证设置}
    \label{fig:e2-s4-see-port}
    %\end{adjustwidth}
\end{figure}



\subsection{步骤5:交换机B创建VLAN20}

在交换机B上创建VLAN20,并将端口0/5划分到VLAN20中.如\autoref{fig:e2-s5-vlan20}

%\usepackage{changepage}
%\usepackage{rotating}
%\usepackage{float}
%\usepackage[section]{placeins}
%\begin{sidewaystable}[!Htp]
\begin{figure}[htp]
    %\begin{adjustwidth}{-1.5cm}{-1cm}
    \centering
    \includegraphics[width=13cm]{"./figure/2018-05-17-22-53-53.png"}
    \caption{交换机B创建VLAN20}
    \label{fig:e2-s5-vlan20}
    %\end{adjustwidth}
\end{figure}



操作完成后,使用\texttt{show vlan id 20}验证实验操作.如\autoref{fig:e2-s5-see-vlan20}

%\usepackage{changepage}
%\usepackage{rotating}
%\usepackage{float}
%\usepackage[section]{placeins}
%\begin{sidewaystable}[!Htp]
\begin{figure}[htp]
    %\begin{adjustwidth}{-1.5cm}{-1cm}
    \centering
    \includegraphics[width=13cm]{"./figure/2018-05-17-22-52-52.png"}
    \caption{在交换机B上查看vlan20}
    \label{fig:e2-s5-see-vlan20}
    %\end{adjustwidth}
\end{figure}


\subsection{步骤6:设置B与A连接的端口模式}


将交换机B上与交换机A相连的端口(假设为端口0/24)定义为\texttt{tag VLAN}模式。设置步骤可见\autoref{fig:e2-s6-port}


%\usepackage{changepage}
%\usepackage{rotating}
%\usepackage{float}
%\usepackage[section]{placeins}
%\begin{sidewaystable}[!Htp]
\begin{figure}[htp]
    %\begin{adjustwidth}{-1.5cm}{-1cm}
    \centering
    \includegraphics[width=13cm]{"./figure/2018-05-17-22-57-16.png"}
    \caption{设置端口模式}
    \label{fig:e2-s6-port}
    %\end{adjustwidth}
\end{figure}


操作完成后,使用\texttt{show interface gig 0/24 switchport}验证设置.见\autoref{fig:e2-s6-see-port}

%\usepackage{changepage}
%\usepackage{rotating}
%\usepackage{float}
%\usepackage[section]{placeins}
%\begin{sidewaystable}[!Htp]
\begin{figure}[htp]
    %\begin{adjustwidth}{-1.5cm}{-1cm}
    \centering
    \includegraphics[width=13cm]{"./figure/2018-05-17-22-55-38.png"}
    \caption{设置端口模式}
    \label{fig:e2-s6-see-port}
    %\end{adjustwidth}
\end{figure}


\subsection{步骤7:测试}

\begin{enumerate}
    \item 测试PC2 与PC3 的连通性,发现PC2与PC3能够ping通。

    %\usepackage{changepage}
    %\usepackage{rotating}
    %\usepackage{float}
    %\usepackage[section]{placeins}
    %\begin{sidewaystable}[!Htp]
    \begin{figure}[htp]
        %\begin{adjustwidth}{-1.5cm}{-1cm}
        \centering
        \includegraphics[width=13cm]{"./figure/id004.png"}
        \caption{}
        \label{fig:}
        %\end{adjustwidth}
    \end{figure}
    

    \item 测试PC1 与 PC2 的连通性
    从图中可以看出,ifconfig指令指示出PC2的IP地址。其后的ping中,与PC3可以连通,与PC1无法连通。
    
    
    %\usepackage{changepage}
    %\usepackage{rotating}
    %\usepackage{float}
    %\usepackage[section]{placeins}
    %\begin{sidewaystable}[!Htp]
        \begin{figure}[htp]
            %\begin{adjustwidth}{-1.5cm}{-1cm}
            \centering
            \includegraphics[width=13cm]{"./figure/2018-05-17-22-58-38.png"}
            \caption{PC1与PC2无法ping通}
            \label{fig:e2-s7-pc1-ping-pc3}
            %\end{adjustwidth}
        \end{figure}
        
    \item 使用\texttt{show ip route}命令查看三层交换机的路由表,并与步骤1比较.如\autoref{fig:e2-s7-route},步骤一中路由表可见\autoref{fig:e2-s1-route}。他们目前都是空的。
    %\usepackage{changepage}
    %\usepackage{rotating}
    %\usepackage{float}
    %\usepackage[section]{placeins}
    %\begin{sidewaystable}[!Htp]
    \begin{figure}[htp]
        %\begin{adjustwidth}{-1.5cm}{-1cm}
        \centering
        \includegraphics[width=13cm]{"./figure/2018-05-17-17-16-01.png"}
        \caption{查看路由表}
        \label{fig:e2-s7-route}
        %\end{adjustwidth}
    \end{figure}
    
    
\end{enumerate}


\subsection{步骤8:设置三层交换机VLAN间的通信}

将交换机A配置成具有路由器的功能,配置不同VLAN接口的地址.

%\usepackage{changepage}
%\usepackage{rotating}
%\usepackage{float}
%\usepackage[section]{placeins}
%\begin{sidewaystable}[!Htp]
\begin{figure}[htp]
    %\begin{adjustwidth}{-1.5cm}{-1cm}
    \centering
    \includegraphics[width=13cm]{"./figure/2018-05-17-17-18-51.png"}
    \caption{设置vlan间通信}
    \label{fig:e2-s8-vlan}
    %\end{adjustwidth}
\end{figure}



%\usepackage{tcolorbox}
%\newtcolorbox{mybox}{}
%\renewtcolorbox{mybox}{colback = red!25!white, colframe = red!75!black}
%\begin{mybox}[title = {}]
\begin{tcolorbox}[title = {讨论}]
虚拟接口VLAN10与虚拟接口VLAN20的IP地址能不能在同一个网段?回答步骤1提出的问题.
\tcblower
不能.
如果在同一个子网中,主机不会尝试将这个数据包发往默认网关,这也就意味着交换机的路由模块无法收到这个数据包.
\end{tcolorbox}


\subsection{步骤9:设置网关}

该步骤设置网关可见步骤一,已经将VLAN 10和VLAN 20内的主机分别将默认网关设为\texttt{192.168.10.254}, \texttt{192.168.20.254}。

\subsection{步骤10:测试是否ping通}

实验测试,使用ping命令查看不同VLAN内的主机能够互相ping通.
启动监控软件Wireshark,互相ping两台计算机并观察.

关于此步骤结果,可见下一节"实验结果"。

\section{实验结果}

\subsection{实验观察}

%\usepackage{tcolorbox}
%\newtcolorbox{mybox}{}
%\renewtcolorbox{mybox}{colback = red!25!white, colframe = red!75!black}
%\begin{mybox}[title = {}]
\begin{tcolorbox}[title = {观察一}]
计算机之间是否连通?
\end{tcolorbox}

能够ping通。

%\usepackage{changepage}
%\usepackage{rotating}
%\usepackage{float}
%\usepackage[section]{placeins}
%\begin{sidewaystable}[!Htp]
\begin{figure}[htp]
    %\begin{adjustwidth}{-1.5cm}{-1cm}
    \centering
    \includegraphics[width=13cm]{"./figure/step_10_success_ping_other_subnet.png"}
    \caption{PC1与PC2,PC3能够ping通}
    \label{fig:e2-s11-ping}
    %\end{adjustwidth}
\end{figure}



%\usepackage{tcolorbox}
%\newtcolorbox{mybox}{}
%\renewtcolorbox{mybox}{colback = red!25!white, colframe = red!75!black}
%\begin{mybox}[title = {}]
\begin{tcolorbox}[title = {观察二}]
能否监控到PC1, PC2, PC3 的ICMP包?
\end{tcolorbox}

能够抓到相应的ICMP包,可见\autoref{fig:e2-catch}.
%\usepackage{changepage}
%\usepackage{rotating}
%\usepackage{float}
%\usepackage[section]{placeins}
%\begin{sidewaystable}[!Htp]
\begin{figure}[htp]
    %\begin{adjustwidth}{-1.5cm}{-1cm}
    \centering
    \includegraphics[width=13cm]{"./figure/id005.png"}
    \caption{查看抓包结果}
    \label{fig:e2-catch}
    %\end{adjustwidth}
\end{figure}


从图中可以看出,主机PC2成功抓到了PC1和PC3的ICMP包。
%\usepackage{tcolorbox}
%\newtcolorbox{mybox}{}
%\renewtcolorbox{mybox}{colback = red!25!white, colframe = red!75!black}
%\begin{mybox}[title = {}]
\begin{tcolorbox}[title = {观察三}]
使用\texttt{show ip route}查看三层交换机的路由表,并与步骤1比较.
\end{tcolorbox}

%\usepackage{changepage}
%\usepackage{rotating}
%\usepackage{float}
%\usepackage[section]{placeins}
%\begin{sidewaystable}[!Htp]
\begin{figure}[htp]
    %\begin{adjustwidth}{-1.5cm}{-1cm}
    \centering
    \includegraphics[width=13cm]{"./figure/2018-05-17-23-09-34.png"}
    \caption{三层交换机路由表}
    \label{fig:e2-s10-route}
    %\end{adjustwidth}
\end{figure}

与拓扑图比较,该路由表中的表项的意思为:

\begin{enumerate}
    \item 表示与该交换机直连着的VLAN相应的子网范围。
    \item 表示该交换机的端口IP。
\end{enumerate}



%\usepackage{tcolorbox}
%\newtcolorbox{mybox}{}
%\renewtcolorbox{mybox}{colback = red!25!white, colframe = red!75!black}
%\begin{mybox}[title = {}]
\begin{tcolorbox}[title = {观察四}]
在命令提示符窗口下,使用route print 是否能够查看实验设置的路由?
\end{tcolorbox}

如\autoref{fig:e2-s11-route}所示,可以在默认网关处看到设置的路由。

%\usepackage{changepage}
%\usepackage{rotating}
%\usepackage{float}
%\usepackage[section]{placeins}
%\begin{sidewaystable}[!Htp]
\begin{figure}[htp]
    %\begin{adjustwidth}{-1.5cm}{-1cm}
    \centering
    \includegraphics[width=13cm]{"./figure/2018-05-18-02-39-50.png"}
    \caption{打印主机路由表}
    \label{fig:e2-s11-route}
    %\end{adjustwidth}
\end{figure}


\section{实验思考}

思考题来自书本与老师提供的PDF材料.

\subsection{思考题一}

\begin{tcolorbox}[title = {思考题一}]
实验用到了三层交换机的路由功能,为什么在VLAN配置好IP地址之后,不同的VLAN间(PC1,PC2)就可以相互通信了?
\end{tcolorbox}

这里的问题是,为什么不需要设置转发表,只需要设置IP地址,不同的VLAN就可以相互通信呢?

这里我们需要理解IP包的转发原理。由前面的图我们可以看到路由器的转发表会自动加上直连的子网和端口IP

\subsection{思考题二}

\begin{tcolorbox}[title = {思考题二}]
请使用\texttt{show ip route} 命令查看三层交换机的路由表,并说明每个条目表示什么?
\end{tcolorbox}


%\usepackage{changepage}
%\usepackage{rotating}
%\usepackage{float}
%\usepackage[section]{placeins}
%\begin{sidewaystable}[!Htp]
\begin{figure}[htp]
    %\begin{adjustwidth}{-1.5cm}{-1cm}
    \centering
    \includegraphics[width=13cm]{"./figure/2018-05-17-17-26-19.png"}
    \caption{}
    \label{fig:}
    %\end{adjustwidth}
\end{figure}




\subsection{思考题三}

%\usepackage{tcolorbox}
%\newtcolorbox{mybox}{}
%\renewtcolorbox{mybox}{colback = red!25!white, colframe = red!75!black}
%\begin{mybox}[title = {}]
\begin{tcolorbox}[title = {思考题三}]
(跨交换机的不同vlan间通信)若要PC1和PC3相互通信,需要怎么进行配置.
\end{tcolorbox}

在这里,经过测试,在前面的配置的基础上,已经可以直接ping通。

在另一种情况中,如果两台交换机各自具有两个不同的VLAN,可以在交换机中各自增加一个与对面VLAN一致的网关,作为一个虚接口,就可以连通了。


%%%%%%%%%%%%%%%%%%%%%%%%%%%%%%%%%%%%%%%%%%%%%%%%%%%%%%%%%%%%%%%%%%%%%%%%%%%%
% 问题三
%%%%%%%%%%%%%%%%%%%%%%%%%%%%%%%%%%%%%%%%%%%%%%%%%%%%%%%%%%%%%%%%%%%%%%%%%%%%


\chapter{问题三}

%\usepackage{tcolorbox}
%\newtcolorbox{mybox}{}
%\renewtcolorbox{mybox}{colback = red!25!white, colframe = red!75!black}
%\begin{mybox}[title = {}]
\begin{tcolorbox}[title = {问题三}]

该部分主要解决这样的问题:
跨交换机不借助trunk模式实现VLAN通信,如何做到?

\end{tcolorbox}


我们使用了以下的方式来实现:

跨交换机实现VLAN通信,而不借助trunk模式,我们可以将两个交换机用两条跳线连接起来,交换机A的24号端口连交换机B的24号端口,A的23号端口连B的23号端口,
然后按照下面的拓扑图,PC1属于交换机A的VLAN10,PC2属于交换机A的VLAN20,PC3属于交换机B的VLAN20,PC4属于交换B的VLAN10,然后再将交换机A和B的24号端口划分为VLAN10,
交换机A和B的23号端口划分为VLAN20。

%\usepackage{changepage}
%\usepackage{rotating}
%\usepackage{float}
%\usepackage[section]{placeins}
%\begin{sidewaystable}[!Htp]
\begin{figure}[htp]
    %\begin{adjustwidth}{-1.5cm}{-1cm}
    \centering
    \includegraphics[width=13cm]{"./figure/2018-05-18-02-58-05.png"}
    \caption{拓扑图}
    \label{fig:e3-topo}
    %\end{adjustwidth}
\end{figure}


下面测试PC3与PC2的互联(它们是属于同一VLAN下,但是在不同交换机,并且没有借助trunk模式):

%\usepackage{changepage}
%\usepackage{rotating}
%\usepackage{float}
%\usepackage[section]{placeins}
%\begin{sidewaystable}[!Htp]
\begin{figure}[htp]
    %\begin{adjustwidth}{-1.5cm}{-1cm}
    \centering
    \includegraphics[width=13cm]{"./figure/6-3x.png"}
    \caption{ping通图}
    \label{fig:e3-ping}
    %\end{adjustwidth}
\end{figure}


结果是,PC3与PC2是可以互联的,因为PC3发出的报文是先在交换机B里面广播给所有属于VLAN20的端口,即就是23号端口,然后交换机B的23号端口会将数据转给交换机A的23号端口,交换机A又会将该报文在自己里面广播给所有属于VLAN20的端口,即就是15号端口,所以PC4会收到来自的PC3的报文,而PC4发送给PC3的报文也是按照这个流程转发给PC3。
  



%%%%%%%%%%%%%%%%%%%%%%%%%%%%%%%%%%%%%%%%%%%%%%%%%%%%%%%%%%%%%%%%%%%%%%%%%%%%
% 参考文献
%%%%%%%%%%%%%%%%%%%%%%%%%%%%%%%%%%%%%%%%%%%%%%%%%%%%%%%%%%%%%%%%%%%%%%%%%%%%
% \cleardoublepage\phantomsection
% \addcontentsline{toc}{chapter}{参考文献}

% \bibliography{opsystem}
% \bibliographystyle{unsrt}

% \begin{thebibliography}{00}
%   \bibitem{r1} 作者. 文章题目 [J].  期刊名, 出版年份,卷号(期数): 起止页码.
%   \bibitem{r2} 作者. 书名 [M]. 版次. 出版地:出版单位,出版年份:起止页码.
%   \bibitem{r3} 邓建松等, 《\LaTeXe~科技排版指南》, 科学出版社.
%   \bibitem{r4} 吴凌云, 《CTeX~FAQ (常见问题集)》, \textit{Version~0.4}, June 21, 2004.
%   \bibitem{r5} Herbert Vo\ss, Mathmode, \url{http://www.tex.ac.uk/ctan/info/math/voss/mathmode/Mathmode.pdf}.
% \end{thebibliography}
%%%%%%%%%%%%%%%%%%%%%%%%%%%%%%%%%%%%%%%%%%%%%%%%%%%%%%%%%%%%%%%%%%%%%%%%%%%%
% 附录
%%%%%%%%%%%%%%%%%%%%%%%%%

\cleardoublepage
\end{document}